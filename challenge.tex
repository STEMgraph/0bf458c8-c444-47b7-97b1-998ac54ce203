\begin{challenge}
    \chatitle{Associative Containers with \texttt{std::map} in C++}
    \begin{chadescription}
    In C++, the \texttt{std::map} class is an associative container that stores elements as key-value pairs, enabling efficient retrieval and organization of data. 
    Each key in a \texttt{std::map} is unique, and the container automatically sorts the elements based on their keys using a comparison function, such as \texttt{std::less} by default.

    The \texttt{std::map} container is particularly useful for scenarios where you need fast access to data based on unique keys, such as dictionaries, databases, or lookup tables. 
    With member functions like \texttt{insert()}, \texttt{find()}, and \texttt{erase()}, it simplifies operations that would otherwise require extensive manual effort in traditional data structures. 

    In this challenge, you will explore how to use \texttt{std::map} to organize and retrieve data efficiently. 
    You will also compare its functionality to that of arrays or custom key-value implementations in C, understanding the advantages it offers in terms of simplicity and performance.
    \end{chadescription}

    \begin{task}
        Write a program using \texttt{std::map} that performs the following steps:
        \begin{enumerate}
            \item Declare an \texttt{std::map} to store student names (strings) as keys and their grades (integers) as values.
            \item Add three entries to the map using \texttt{insert()}.
            \item Retrieve and print the grade of a specific student using \texttt{find()}.
            \item Print all the entries in the map using a range-based \texttt{for} loop.
        \end{enumerate}

        Save your program and run it to ensure the output matches your expectations.

        \begin{questions}
            \item What happens if you try to insert an entry with a key that already exists in the map?
            \item How does \texttt{find()} simplify searching for a key compared to a custom key-value implementation in C?
            \item Why is the automatic sorting of keys useful when iterating over a map?
        \end{questions}
    \end{task}

    \begin{task}
        Write a program using \texttt{std::map} that simulates a product price catalog:
        \begin{enumerate}
            \item Declare an \texttt{std::map} to store product names (strings) as keys and their prices (doubles) as values.
            \item Add five products to the map using the subscript operator (\texttt{[]}) to insert key-value pairs.
            \item Update the price of one product using the subscript operator.
            \item Remove a product from the catalog using \texttt{erase()}.
            \item Print the entire catalog after each operation.
        \end{enumerate}

        Save your program, compile it, and ensure it runs correctly.

        \begin{questions}
            \item What are the advantages of using the subscript operator (\texttt{[]}) for insertion and modification in \texttt{std::map}?
            \item How does \texttt{erase()} ensure that only the specified key-value pair is removed from the map?
            \item What would happen if you access a key in the map that doesn’t exist using the subscript operator?
        \end{questions}
    \end{task}

    \begin{task}
        Write a program using \texttt{std::map} to count the frequency of words in a sentence:
        \begin{enumerate}
            \item Read a sentence from the user and split it into words (assume words are separated by spaces).
            \item Use an \texttt{std::map} to store each word as a key and its frequency as the value.
            \item Increment the frequency count for each word as it appears in the sentence.
            \item Print the word frequency table, showing each word and its count.
        \end{enumerate}

        Save your program, compile it, and ensure it runs correctly. This task requires careful use of the subscript operator to manage the frequency counts.

        \begin{questions}
            \item Why is \texttt{std::map} a good choice for counting word frequencies compared to an array or a custom implementation?
            \item How does \texttt{std::map}'s automatic key management help simplify the process of updating word counts?
            \item What would happen if the same sentence contained uppercase and lowercase versions of a word? How could you address this issue?
        \end{questions}
    \end{task}

    \begin{advise}
        Working with \texttt{std::map} helps you understand how associative containers simplify the management of key-value pairs in C++. 
        Unlike manual implementations in C, \texttt{std::map} provides built-in methods for insertion, retrieval, and deletion, making your code more efficient and easier to maintain. 
        Through this challenge, you have explored scenarios where \texttt{std::map} shines, such as cataloging, searching, and counting. 
        Always remember that \texttt{std::map} automatically sorts keys and ensures uniqueness, which is invaluable for many applications. 
        With practice, you'll find \texttt{std::map} to be an essential tool for organizing and accessing data efficiently.
    \end{advise}
\end{challenge}
